\documentclass[10pt, a4paper, draftcls, conference, onecolumn]{IEEEtran}
\IEEEoverridecommandlockouts
% The preceding line is only needed to identify funding in the first footnote. If that is unneeded, please comment it out.
\usepackage[T1]{fontenc}
\usepackage{lmodern}
\usepackage{cite}
\usepackage{IMTtikz}
\usepackage[hidelinks,colorlinks=true,urlcolor=blue,linkcolor=black]{hyperref}
\usepackage{algorithmic}
\usepackage{graphicx}
\usepackage{textcomp}
\usepackage{xcolor}
\def\BibTeX{{\rm B\kern-.05em{\sc i\kern-.025em b}\kern-.08em
    T\kern-.1667em\lower.7ex\hbox{E}\kern-.125emX}}

\usepackage[stretch=10,shrink=10]{microtype}
\AtBeginEnvironment{verbatim}{\microtypesetup{activate=false}}
\usepackage{polyglossia}

\setdefaultlanguage[variant=brazilian]{portuguese}
\setotherlanguages{english}
\SetLanguageKeys{portuguese}{indentfirst=false}
\SetLanguageKeys{english}{indentfirst=false}

\begin{document}

\title{\textbf{Projeto de Pesquisa para Iniciação Científica} \\
{Comparando a utilização de soluções livres e proprietárias para programação geral de GPUs}
% \thanks{Identify applicable funding agency here. If none, delete this.}
}

\author{\IEEEauthorblockN{Isabella Basso do Amaral} \\
\and
\IEEEauthorblockN{Orientador: Alfredo Goldman}
\IEEEauthorblockA{\textit{Instituto de Matemática e Estatísica} \\
\textit{Universidade de São Paulo}\\
São Paulo, Brasil \\
\texttt{\href{mailto:gold@ime.usp}{\nolinkurl{gold@ime.usp}}}}
}

\maketitle

\begin{abstract}
    A adoção, por parte da comunidade científica, de \textit{software} com
    intuito de aproximar soluções para problemas relevantes no cenário
    contemporâneo traz consigo desafios além dos técnicos, principalmente, no
    que diz respeito à utilização de soluções livres para a execução de tais
    pesquisas. Esta é essencial para que não violemos princípios básicos da
    pesquisa científica como a entendemos nos dias de hoje, como por exemplo
    o princípio da reprodutibilidade. No presente projeto, então, verificamos a
    distribuição de pesquisas em relação à sua utilização de soluções livres ou
    proprietárias, especificamente no contexto de \textit{GPU General
    Programming} (GPGPU), assim como as disparidades entre opções livres e
    proprietárias que dificultam o cenário ideal. A partir deste ponto,
    exploramos os passos necessários para facilitar a adoção de soluções
    livres, assim como o seu desenvolvimento.
\end{abstract}

\begin{IEEEkeywords}
    GPU, CUDA, Vulkan, Compute, OpenCL, software, livre, aberto, proprietário
\end{IEEEkeywords}

\section{Introdução}

O computador tornou-se parte indispensável da pesquisa acadêmica, seja
facilitando a leitura de artigos e livros ou ``resolvendo'' (aproximando)
problemas numéricos, os quais são, em grande maioria, insolúveis sob o olhar
analítico.

A aproximação de soluções para problemas numéricos em \textit{software}, embora
não seja uma questão nova ou sequer teórica, demanda a utilização de técnicas
avançadas de programação exigindo, por vezes, o uso de \textit{software}
extremamente particular, complexo e que demanda conhecimento técnico
específico, tanto da teoria quanto do \textit{software} em questão.

\subsection{Otimização}

Também no que diz respeito aos problemas numéricos, gostaríamos de ressaltar,
em especial, a possibilidade de paralelizá-los, agilizando (por vezes
desproporcionalmente) a produção de resultados. Em diversos casos, tais
otimizações são necessárias para que possamos produzir resultados
oportunamente, levando em conta simplesmente a capacidade de paralelização do
\textit{hardware} moderno \textit{versus} suas capacidades de execução linear,
notamos que frequentemente ele é pelo menos milhares de vezes mais capaz na
primeira categoria do que na segunda.

A GPU (\textit{Graphics Processing Unit}) é um componente especializado em
tarefas de paralelização, de tal forma que, atualmente, é utilizada
extensivamente para auxiliar na resolução de problemas nas áreas de física,
biológia, química e até mesmo da matemática moderna sendo, por vezes, mais
relevante do que a CPU (\textit{Central Processing Unit}) de um dado sistema em
aplicações como aquelas da ciência de dados.

\subsection{Sobre o uso de \textit{software} livre em aplicações científicas e sua importância}\label{sec:soft-livre}

No que tange a utilização de \textit{software} para aplicações científicas, nos
preocupamos principalmente com a utilização de \textit{software} livre, que,
estritamente falando, é aquele que se alinha perfeitamente com os princípios da
pesquisa científica.

Chamamos de \textbf{\textit{software} livre} aquele que é aberto para ser lido
e auditado, e que não depende de corporações (embora possa ser auxiliado por
estas) para que se mantenha. Possuíndo comunidades autonomas de usuários e
desenvolvedores que o mantém de acordo com interesses pessoais e plurais, porém
não necessariamente sem o envolvimento de capital.

O \textit{software} livre permeia todo o contexto computacional moderno, sendo
utilizado extensivamente na internet (e.g. $\approx 80$ dos servidores utilizam
\textit{Linux} \cite{w3techs}), no contexto do desenvolvimento de \textit{software} e,
especialmente, na pesquisa científica.

Contrastamos tal com o \textbf{\textit{software} proprietário}, onde a
contribuição de um indivíduo sempre será atrelada com interesses corporativos
pois pertence à uma empresa e é secreto sendo, portanto, impossível auditá-lo.

Existe ainda o \textit{superset} de \textit{software} livre, que é o
\textbf{\textit{software} aberto}, onde pode existir uma empresa que o mantém
de acordo com seus interesses, porém este não é secreto, podendo ser auditado e
aceitando contribuições individuais\footnote{ No presente projeto, no entanto,
vamos nos dedicar exclusivamente à problemática do \textit{software} livre X
\textit{software} proprietário. }.

Note, então, que a utilização de \textit{software} proprietário não cabe no
contexto científico, pois demanda desmedida confiança à interesses
corporativos, indo totalmente de encontro a princípios fundacionais da ciência
como a temos hoje, onde reprodutibilidade e transparência são essenciais.

Com isso em mente, advogamos pelo uso exclusivo do \textit{software} livre
nesse contexto o que, infelizmente, ainda não é totalmente plausível como será
explorado mais a frente.

\section{Justificativa}

Haja vista que a GPU é um componente de \textit{hardware} sendo, então,
necessário comandá-la de forma específica e precisa, idealmente gostaríamos de
abstrair detalhes do \textit{hardware} específico, de tal forma que tenhamos
fino controle sobre sua utilização, sem comprometer, porém, seu desempenho
(i.e. a abstração deve otimizar o que quer que seja tendo em mente o
\textit{hardware} específico).

\subsection{Compiladores}

Classicamente, utilizamos \textit{shaders} para a programação de GPUs, sendo
mais ou menos análogos à programação de CPUs em sua sintaxe, mas não
necessariamente em sua lógica, o que torna seu uso não-trivial para alguém sem
o conhecimento específico. O código do \textit{shader} é compilado em
etapas, sendo primeiramente convertido para uma linguagem intermediária, onde é
otimizado com base em conceitos primitivos próprios para abordagens de
paralelização. Depois, sendo convertido para um binário otimizado para a GPU
específica onde será executado, e então enviado para o \textit{hardware}.

\subsection{Comparando o \textit{software} livre com o proprietário para aplicações científicas}

No contexto específico de aplicações numéricas, no presente projeto nosso
interesse é voltado à \textbf{programação geral} com GPUs (GPGPU), onde temos
utilização extensiva da linguagem \textbf{CUDA}, desenvolvida pela empresa
NVIDIA -- uma pioneira em computação gráfica. Dada sua simplicidade e
abrangência, o CUDA segue invicto em aplicações científicas que utilizam GPUs.

Gostaríamos de notar, no entanto, que essa é uma tecnologia proprietária, e que
sua adoção supera a de alternativas livres por uma grande margem sendo, então,
de grande importância para a comunidade científica contemporânea que existam
alternativas viáveis e livres para essas aplicações, como dito em \ref{sec:soft-livre}.

% \begin{figure}[h]
%     \centering
%     \includegraphics[width=0.5\textwidth]{figures/cuda.png}
%     \caption{Comparação entre o \textit{software} livre e o proprietário para aplicações científicas}
%     \label{fig:cuda}
% \end{figure}

\section{Objetivos}

Dada a utilização desproporcional do CUDA em relação à alternativas livres,
possuímos diversos questionamentos que devem ser abordados no presente
projeto, como:

\begin{enumerate}
    \item Qual a figura exata da adoção de \textit{software} livre para aplicações científicas?
    \item Onde o CUDA se sobressai em comparação à essas alternativas?
    \item Como podemos diminuir essa diferença e como facilitar a migração para soluções que usam exclusivamente código livre?
    \item O cenário ideal é factível?
\end{enumerate}

Como alternativas (livres) ao CUDA, destacamos o \textbf{OpenCL} e o
\textbf{Vulkan}, os quais serão utilizados durante o projeto para
aperfeiçoamento e comparações.

% \begin{figure}[htbp]
% \centerline{\includegraphics{fig1.png}}
% \caption{Example of a figure caption.}
% \label{fig}
% \end{figure}

\section{Metodologia}

\section{Cronograma}

% \begin{table}
%     \centering
%     \begin{tabular}{@}
        
%     \end{tabular}
% \end{table}

\bibliographystyle{IEEEtran}
\bibliography{IEEEabrv,references}

\end{document}
