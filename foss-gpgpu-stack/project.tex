\documentclass[10pt, conference]{IEEEtran}
% \IEEEoverridecommandlockouts
% The preceding line is only needed to identify funding in the first footnote. If that is unneeded, please comment it out.
\usepackage[T1]{fontenc}
\usepackage{lmodern}
\usepackage{cite}
\usepackage{IMTtikz}
\usepackage[hidelinks,colorlinks=true,urlcolor=blue,linkcolor=black]{hyperref}
\usepackage{algorithmic}
\usepackage{graphicx}
\usepackage{textcomp}
\usepackage{xcolor}
\def\BibTeX{{\rm B\kern-.05em{\sc i\kern-.025em b}\kern-.08em
    T\kern-.1667em\lower.7ex\hbox{E}\kern-.125emX}}

\usepackage[stretch=10,shrink=10]{microtype}
\AtBeginEnvironment{verbatim}{\microtypesetup{activate=false}}
\usepackage{polyglossia}

\setdefaultlanguage[variant=brazilian]{portuguese}
\setotherlanguages{english}
\SetLanguageKeys{portuguese}{indentfirst=false}
\SetLanguageKeys{english}{indentfirst=false}

\begin{document}

\title{\textbf{Projeto de Pesquisa para Iniciação Científica} \\
{Comparando a utilização de soluções livres e proprietárias para programação geral de GPUs}
% \thanks{Identify applicable funding agency here. If none, delete this.}
}

\author{\IEEEauthorblockN{Isabella Basso do Amaral}
\and
\IEEEauthorblockN{Orientador: Alfredo Goldman}
\IEEEauthorblockA{\textit{Instituto de Matemática e Estatísica} \\
\textit{Universidade de São Paulo}\\
São Paulo, Brasil \\
\texttt{\href{mailto:gold@ime.usp}{\nolinkurl{gold@ime.usp}}}}
}

\maketitle

\begin{abstract}
    meow meow meow
\end{abstract}

\begin{IEEEkeywords}
    GPU, CUDA, Vulkan, Compute, OpenCL, software, livre, aberto, proprietário
\end{IEEEkeywords}

\section{Introdução}

O computador tornou-se parte indispensável da pesquisa acadêmica, tendo
funcionalidades que abrangem desde facilitar a leituras de artigos e livros até
a resolução (aproximada) de problemas numéricos, que seriam insolúveis
analiticamente.

Quanto ao último exemplo, gostaríamos de ressaltar, em especial, a capacidade
de paralelizar tais problemas numéricos, de tal forma que conseguimos reduzir o
tempo de execução. Otimizações no tratamento desses problemas são uma
necessidade para o desenvolvimento da pesquisa científica como se dá atualmente.

\subsection{A GPU e sua importância na pesquisa contemporânea}

Um componente especializado em paralelização é a GPU (\textit{Graphics
Processing Unit}), ela vem sendo extensivamente utilizada para auxiliar na
resolução de problemas físicos, biológicos, químicos e até mesmo da matemática
moderna, tornando-se, por vezes, mais importante do que a CPU (\textit{Central
Processing Unit}) do dito sistema.

Haja vista que a GPU é um componente de hardware, o que significa que é
necessário comandá-la, idealmente abstraíndo detalhes do hardware específico,
de tal forma que tenhamos fino controle sobre sua utilização porém sem
comprometer seu desempenho (i.e. a abstração deve otimizar o que quer que seja
tendo em mente o hardware específico).

\subsection{Compiladores}

Classicamente, utilizamos \textit{shaders} para a programação de GPUs, sendo
mais ou menos análogos à programação de CPU. O código do \textit{shader} é,
então, compilado em etapas, sendo primeiramente convertido para uma linguagem
intermediária, onde é otimizado com base em conceitos primitivos próprios para
abordagens de paralelização, como o \textit{thread} ou o \textit{block}.
Depois, sendo convertido para um binário otimizado para a GPU específica onde
será executado, e então enviado para o \textit{hardware}.

\subsection{A relevância do \textit{software} livre no contexto científico}

Chamamos de \textbf{\textit{software} livre} aquele que não depende de
corporações (embora possa ser auxiliado por estas) para que se mantenha.
Possuíndo comunidades autonomas de usuários e desenvolvedores que o mantém
de acordo com interesses pessoais e plurais, porém não necessariamente sem
o envolvimento de capital.

O \textit{software} livre permeia todo o contexto computacional moderno,
sendo utilizado extensivamente na internet (e.g. $\approx X$ dos
servidores utilizam \textit{Linux}), no contexto do desenvolvimento de
\textit{software} e, especialmente, na pesquisa científica.

Contrastamos tal com o \textbf{\textit{software} proprietário}, onde a
contribuição de um indivíduo sempre será atrelada com interesses corporativos
pois pertence à uma empresa e é secreto sendo, portanto, impossível auditá-lo.

Existe ainda o \textit{superset} de \textit{software} livre, que é o
\textbf{\textit{software} aberto}, onde pode existir uma empresa que o mantém
de acordo com seus interesses, porém este não é secreto, podendo ser auditado
e aceitando contribuições individuais\footnote{
    No presente projeto, no entanto, dedicaremos-nos exclusivamente à
    problemática do \textit{software} livre X \textit{software} proprietário.
}.

Note, então, que a utilização de \textit{software} proprietário não cabe no
contexto científico, pois demanda desmedida confiança à interesses corporativos,
indo totalmente de encontro a princípios fundacionais da ciência como a temos
hoje, onde reprodutibilidade e transparência são essenciais.

Com isso em mente, advogamos pelo uso exclusivo do \textit{software} livre nesse
contexto o que, infelizmente, ainda não é totalmente plausível.

\subsection{Comparando o \textit{software} livre com o proprietário para aplicações científicas}

Como dito anteriormente, utilizamos \textit{shaders} para a programação de GPUs,
e no contexto específico de aplicações numéricas, nosso interesse é voltado à
\textbf{programação geral} com GPUs (GPGPU), onde temos utilização extensiva
da linguagem \textbf{CUDA}, desenvolvida pela empresa NVIDIA.

Dada sua simplicidade e abrangência, o CUDA segue invicto em aplicações
científicas, superando alternativas livres por uma grande margem.

% \begin{figure}[h]
%     \centering
%     \includegraphics[width=0.5\textwidth]{figures/cuda.png}
%     \caption{Comparação entre o \textit{software} livre e o proprietário para aplicações científicas}
%     \label{fig:cuda}
% \end{figure}

Esse fato nos traz diversos questionamentos que devem ser abordados no presente
projeto, como:

\begin{enumerate}
    \item Onde o CUDA se sobressai em comparação às alternativas livres?
    \item Como podemos diminuir essa diferença e facilitar a adoção de alternativas livres?
    \item Quais são as dificuldades para que a migração ocorra e como podemos facilitá-la?
\end{enumerate}

Como alternativas ao CUDA, destacamos o \textbf{OpenCL} e o \textbf{Vulkan},
que serão utilizados nas comparações.

% \begin{figure}[htbp]
% \centerline{\includegraphics{fig1.png}}
% \caption{Example of a figure caption.}
% \label{fig}
% \end{figure}

\begin{thebibliography}{00}
\bibitem{b1} G. Eason, B. Noble, and I. N. Sneddon, ``On certain integrals of Lipschitz-Hankel type involving products of Bessel functions,'' Phil. Trans. Roy. Soc. London, vol. A247, pp. 529--551, April 1955.
\bibitem{b2} J. Clerk Maxwell, A Treatise on Electricity and Magnetism, 3rd ed., vol. 2. Oxford: Clarendon, 1892, pp.68--73.
\bibitem{b3} I. S. Jacobs and C. P. Bean, ``Fine particles, thin films and exchange anisotropy,'' in Magnetism, vol. III, G. T. Rado and H. Suhl, Eds. New York: Academic, 1963, pp. 271--350.
\bibitem{b4} K. Elissa, ``Title of paper if known,'' unpublished.
\bibitem{b5} R. Nicole, ``Title of paper with only first word capitalized,'' J. Name Stand. Abbrev., in press.
\bibitem{b6} Y. Yorozu, M. Hirano, K. Oka, and Y. Tagawa, ``Electron spectroscopy studies on magneto-optical media and plastic substrate interface,'' IEEE Transl. J. Magn. Japan, vol. 2, pp. 740--741, August 1987 [Digests 9th Annual Conf. Magnetics Japan, p. 301, 1982].
\bibitem{b7} M. Young, The Technical Writer's Handbook. Mill Valley, CA: University Science, 1989.
\end{thebibliography}
\vspace{12pt}
\color{red}
IEEE conference templates contain guidance text for composing and formatting conference papers. Please ensure that all template text is removed from your conference paper prior to submission to the conference. Failure to remove the template text from your paper may result in your paper not being published.

\end{document}
