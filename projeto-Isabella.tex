\documentclass[12pt,reqno]{amsart}



\setcounter{tocdepth}{4}
\setcounter{secnumdepth}{4}

\usepackage{psfrag}
\usepackage{graphicx}
\usepackage[english,brazil]{babel}
\usepackage[utf8]{inputenc}
\usepackage{textcomp}
\usepackage[T1]{fontenc}
\usepackage{lmodern}
\usepackage{fullpage}
\usepackage{pdfpages}
%----------------------------
\usepackage{eurosym}

\usepackage{color}
\usepackage{geometry}
%
\geometry{left=25mm,right=25mm, top=23mm, bottom=25mm}

%\usepackage[colorlinks,backref]{hyperref}
\usepackage[]{hyperref}
\usepackage{srcltx}
\usepackage{multicol}
\usepackage{array}

%\setcounter{page}{-2}

\makeatletter
\renewcommand{\labelenumi}{\theenumi.}
\makeatother

%%%%%%%%%%%%%%%%%%%%%%%%%%%%%%%%%%%%%%%%%%%%%%%%%%%%%%%%%%%%%%%%%%%%%%
\let\eps\varepsilon
%%%%%%%%%%%%%%%%%%%%%%%%%%%%%%%%%%%%%%%%%%%%%%%%%%%%%%%%%%%%%%%%%%%%%%

\newtheoremstyle{note}% name
  {3pt}%      Space above
  {3pt}%      Space below 
  {\sl}%      Body font
  {}%         Indent amount (empty = no indent, \parindent = para indent)
  {\bfseries}% Thm head font
  {.}%        Punctuation after thm head
  {.5em}%     Space after thm head: " " = normal interword space;
                                %       \newline = linebreak
  {}%         Thm head spec (can be left empty, meaning `normal')

  
  
\def\tand{\ \text{and}\ }
\def\qand{\quad\text{and}\quad}
\def\qqand{\qquad\text{and}\qquad}

\let\emptyset=\varnothing
\let\setminus=\smallsetminus
\let\backslash=\smallsetminus
\let\subset\subseteq
\let\log=\ln
\let\phi=\varphi

\newcommand{\Phibad}{\Phi_{\textsc{bad}}}
%\homit make ``hom'' italic in definitions, lemmas and theorems.

\newtheorem{teorema}             {Teorema}       
\newtheorem{Afirmativa}[teorema] {Afirmativa}         
\newtheorem{lema}      [teorema] {Lema}         
\newtheorem{corolario} [teorema] {Corolário}     
\newtheorem{fato}      [teorema] {Fato}          
\newtheorem{conjectura}[teorema] {Conjectura}    
\newtheorem{problema}  [teorema] {Problema}       

\theoremstyle{note}
\newtheorem{propriedade}  [teorema] {Propriedade}       
\newtheorem{definicao}  [teorema] {Definição}       
\newtheorem{exercicio}  [teorema] {Exercício}       


\usepackage{setspace}
\onehalfspacing
\renewcommand{\baselinestretch}{1.2}

\begin{document}



\title[]{Tópicos clássicos e contemporâneos em\\ Combinatória Extremal\\
\ \\ Projeto de iniciação científica}

 \thanks{isabellabdoamaral@usp.br, mota@ime.usp.br}
%   
\maketitle
\thispagestyle{empty}


\indent{\bf Aluna:} Isabella Basso do Amaral\\
\indent{\bf Orientador:} Guilherme Oliveira Mota


{\selectlanguage{brazil}
\begin{abstract}
Este é o projeto para a iniciação científica de Isabella Basso do
Amaral, a ser desenvolvido sob a supervisão de Guilherme Oliveira
Mota.
Propomos a investigação de tópicos importantes da Combinatória
Extremal, desde os clássicos resultados até importantes técnicas recentes.
\end{abstract}


\section{Introdução}

Temos visto nos últimos anos um crescente desenvolvimento na área de
Combinatória, em particular em Combinatória Extremal, que é a parte da
Combinatória dedicada a investigar o quão grande (ou pequeno) um
conjunto de estruturas matemáticas pode ser, quando consideramos um
grupo de restrições.
Além de clássicos teoremas que continuam sendo utilizados com sucesso
na resolução de diversos problemas, novos métodos surgidos nos últimos
anos vêm permitindo avanços importantes em problemas que antes eram
considerados de difícil abordagem.
Como exemplo, podemos citar o \emph{Método de Containers em
  Hipergrafos}, desenvolvido a apenas alguns anos atrás, de forma
independente, por Saxton e Thomasson~\cite{SaTh15} e por Balogh,
Morris e Samotij~\cite{BaMoSa15}. 


Neste projeto propomos o estudo avançado de resultados e técnicas
clássicas de Combinatória, bem como a investigação das estratégias
modernas surgidas ao longo das últimas décadas.
Para isso, a aluna estudará principalmente tópicos selecionados
de~\cite{Ju11} e os capítulos sobre Combinatória
de~\cite{AiZi14}, que contém provas elegantes e concisas de
resultados importantes da área.
Além disso, serão estudados alguns capítulos de~\cite{Di18} para
complementar a pesquisa.
Mencionamos ainda que o livro de Bona~\cite{Bo17} será útil para
dirimir dúvidas conceituais sobre Teoria de Grafos que surjam ao longo
da execução do projeto.

A metodologia a ser utilizada é padrão na área.
Por ser uma pesquisa de cunho teórico, consistirá do
estudo da bibliografia combinada com reuniões semanais com o
orientador e outros alunos.
As reuniões ocorrerão em conjunto com Yoshiharu Kohayakawa
(IME--USP), colaborador frequente do orientador.
Além disso, outros alunos de Iniciação Científica do grupo de pesquisa
participarão dessas reuniões,
permitindo que a aluna exponha os avanços obtidos e tenha contato com
a pesquisa realizada pelos outros alunos, gerando um ambiente propício
para o amadurecimento intelectual de todos os participantes. 

Ao final do projeto, a aluna terá adquirido a base matemática
necessária para estudar artigos científicos relevantes e trabalhar na
solução de problemas em aberto na área.


\bibliographystyle{amsplain}
\bibliography{bibliografia}

\end{document}
